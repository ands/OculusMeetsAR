\begin{center}
\textbf{Developed by Andreas Mantler}
\end{center}
ARLibOgre provides an engine-dependent integration of ARLib into the \emph{Object oriented Graphics Rendering Engine (OGRE)}\cite{ogre}.

\section{Example}\label{example}

Usage in an OGRE scene (have a look at the
\href{https://github.com/ands/OculusMeetsAR/tree/master/ARLib_Samples}{ARLib\_Samples}
for more details):

\begin{lstlisting}
// create a rift node and add it to the tracking system
riftNode = new ARLib::RiftSceneNode(
    rift, sceneManager, 0.001f, 50.0f, 1);
tracker->addRigidBodyEventListener(riftNode);

// add a render target that renders to the rift screen (window)
renderTarget = new ARLib::RiftRenderTarget(rift, root, window);
riftNode->addRenderTarget(renderTarget);

// add a render target that renders to an additional debug output
smallRenderTarget = new ARLib::DebugRenderTarget(smallWindow);
riftNode->addRenderTarget(smallRenderTarget);

// attach video screens (call riftVideoScreens->update() each frame!)
riftVideoScreens = new ARLib::RiftVideoScreens(sceneManager,
    riftNode, leftVideoPlayer, rightVideoPlayer, tracker);
\end{lstlisting}

\section{Class Listing}
ARLibOgre is comprised of the following classes:

\newpage
\subsubsection{\texorpdfstring{\href{https://github.com/ands/OculusMeetsAR/blob/master/ARLib/include/ARLib/Ogre/RiftSceneNode.h}{ARLib::RiftSceneNode}
: public
\href{https://github.com/ands/OculusMeetsAR/blob/master/ARLib/include/ARLib/Tracking/RigidBodyEventListener.h\#L33}{ARLib::RiftRigidBodyEventListener}}{ARLib::RiftSceneNode : public ARLib::RiftRigidBodyEventListener}}\label{arlibriftscenenode-public-arlibriftrigidbodyeventlistener}

The \texttt{RiftSceneNode} represents the virtual Oculus Rift device in the OGRE
scene. Especially, it contains the two cameras which represent the eyes.
\texttt{RenderTargets} can be connected to the \texttt{RiftSceneNode}.

\begin{lstlisting}
// constructor
RiftSceneNode(ARLib::Rift *rift, Ogre::SceneManager *sceneManager,
    float zNear, float zFar, unsigned int rigidBodyID);

// the various parts the RiftSceneNode is comprised of
ARLib::Rift *getRift();
Ogre::SceneNode *getBodyNode();
Ogre::SceneNode *getHeadNode();
Ogre::SceneNode *getHeadCalibrationNode();
Ogre::Camera *getLeftCamera();
Ogre::Camera *getRightCamera();

// manually adjustable orientation
// useful for mouselook if no actual Oculus Rift device is connected
void setPitch(Ogre::Radian angle);
void setYaw(Ogre::Radian angle);

// maintain the render targets
void addRenderTarget(ARLib::RenderTarget *renderTarget);
void removeRenderTarget(ARLib::RenderTarget *renderTarget);
void removeAllRenderTargets();
\end{lstlisting}

\subsubsection{\texorpdfstring{\href{https://github.com/ands/OculusMeetsAR/blob/master/ARLib/include/ARLib/Ogre/RiftRenderTarget.h}{ARLib::RiftRenderTarget}
: public
\href{https://github.com/ands/OculusMeetsAR/blob/master/ARLib/include/ARLib/Ogre/RenderTarget.h}{ARLib::RenderTarget}}{ARLib::RiftRenderTarget : public ARLib::RenderTarget}}\label{arlibriftrendertarget-public-arlibrendertarget}

The \texttt{RiftRenderTarget} is the \texttt{RenderTarget} that renders to the Oculus Rift
screen (in extended desktop mode), using the provided distortion meshes,
vignetting and chromatic aberration correction.

\begin{lstlisting}
// constructor
RiftRenderTarget(ARLib::Rift *rift, Ogre::Root *root,
    Ogre::RenderWindow *renderWindow);
\end{lstlisting}

\subsubsection{\texorpdfstring{\href{https://github.com/ands/OculusMeetsAR/blob/master/ARLib/include/ARLib/Ogre/DebugRenderTarget.h}{ARLib::DebugRenderTarget}
: public
\href{https://github.com/ands/OculusMeetsAR/blob/master/ARLib/include/ARLib/Ogre/RenderTarget.h}{ARLib::RenderTarget}}{ARLib::DebugRenderTarget : public ARLib::RenderTarget}}\label{arlibdebugrendertarget-public-arlibrendertarget}

The \texttt{DebugRenderTarget} is a \texttt{RenderTarget} that renders both camera images
next to each other directly to a debug output window. Use this, if you
don't have an actual Oculus Rift connected.

\begin{lstlisting}
// constructor
DebugRenderTarget(Ogre::RenderWindow *renderWindow);
\end{lstlisting}

\subsubsection{\texorpdfstring{\href{https://github.com/ands/OculusMeetsAR/blob/master/ARLib/include/ARLib/Ogre/RiftVideoScreens.h}{ARLib::RiftVideoScreens}}{ARLib::RiftVideoScreens}}\label{arlibriftvideoscreens}

\texttt{RiftVideoScreens} manages the streaming and placement of camera images.
The camera images are placed at the location in the scene where they
have been captured (according to their capture timestamp and the Rifts'
tracking data history).

\begin{lstlisting}
// constructor
RiftVideoScreens(
    Ogre::SceneManager *sceneManager, ARLib::RiftSceneNode *riftNode,
    ARLib::VideoPlayer *videoPlayerLeft,
    ARLib::VideoPlayer *videoPlayerRight,
    ARLib::TrackingManager *trackingManager = NULL);

// use these to fine-tune the video image positions and sizes
void setOffsets(Ogre::Vector2 leftOffset, Ogre::Vector2 rightOffset);
void setScalings(Ogre::Vector2 leftScale, Ogre::Vector2 rightScale);

// call this every frame to upload the latest camera images
void update();
\end{lstlisting}