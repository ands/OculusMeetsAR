\begin{center}
\textbf{Developed by Tim Michels}
\end{center}
\begin{enumerate}
\item 
Before starting the stereo calibration, download the \href{https://sites.google.com/site/scarabotix/ocamcalib-toolbox}{OCamCalib Toolbox} for Matlab provided by Davide Scaramuzza\cite{ocamcalib}. Use this tool to calculate the intrinsic camera parameters for both cameras. We recommend our calibration tool to take snapshots for both cameras (see 3). Rename the two resulting calib\_results.txt to calib\_results\_CAM1.txt (left camera) and calib\_results\_CAM2.txt (right camera) and place these files at /Hardware/Calibration\_Tool/media/
\item
  Connect your webcams to the system and start our
  \href{https://github.com/ands/OculusMeetsAR/tree/master/Hardware/Calibration_Tool}{calibration
  tool}. You will see the main menu.
	
  \includegraphics*[width=.92\textwidth]{0.jpg}
	
  Note: We built our system for two Logitech C310 webcams with Logitech
  drivers installed. If you don't want to use Logitech Cxxx cameras or
  the Logitech drivers, you will have to modify the
  \href{https://github.com/ands/OculusMeetsAR/tree/master/ARLib/src/Video}{VideoPlayer.cpp}.
\item
  Take snapshots, which you can use for the stereo calibration (main
  menu option 1). After selecting this option, press SPACE to take
  snapshot or ESC to exit.
\item
  Load the images from the media folder (main menu option 2). The tool
  will list the snapshots.
	
  \includegraphics*[width=.92\textwidth]{2.jpg}
	
  Now select the snapshots, you want to use for calibration (simply
  press ENTER to select all). Here we decided to choose the snapshot
  pairs 1,2,3,4 and 7. Then our tool loads and undistorts (according to
  the previously determined ocam models) the images.
	
  \includegraphics*[width=.92\textwidth]{3.jpg}
	
\item
  Extract and match keypoints (main menu option 3).
	
  \includegraphics*[width=.92\textwidth]{4.jpg}
	
  Select the second option and choose the method for keypoint extraction
  and matching -
  \href{http://en.wikipedia.org/wiki/Scale-invariant_feature_transform}{SIFT}\cite{sift},
  chessboard detection (using OpenCV's chessboard corner detector), or a
  combination of both.
	
  \includegraphics*[width=.92\textwidth]{5.jpg}
	
  If you choose the first or last method, you will have the option to
  check the matches manually.
	
  \includegraphics*[width=.92\textwidth]{6.jpg}
	
  If you want to examine the matches manually, for every match you will
  get an image like this.
	
  \includegraphics*[width=.92\textwidth]{7.jpg}
	
  Now press the RIGHT ARROW KEY to accept the currently displayed match
  or any other key to reject it. After matching the keypoints of an
  image pair, the console will tell you, how many matches you accepted
  and how many of those were added to the final collection of matches.
  Some correct matches might be discarded because of previously added
  very similar matches.
	
  \includegraphics*[width=.92\textwidth]{8.jpg}
	
  For chessboard matching you need to enter the number of inner
  chessboard corners (e.g.~for a standard 8x8 chessboard enter 7 as
  x-size as well as y-size).
	
  \includegraphics*[width=.92\textwidth]{9_0.jpg}
	
\item
  Approximate the
  \href{http://en.wikipedia.org/wiki/Epipolar_geometry}{epipolar
  geometry}\cite{epipolar} (main menu option 4). With the calculated or selected
  matches we now want to estimate the fundamental matrix as well as the
  homographies, which we can use to rectify our camera streams.
	
  \includegraphics*[width=.92\textwidth]{9_1.jpg}
	
  Here we used two methods: 1. The implementation of Hartley's
  \href{http://docs.opencv.org/modules/calib3d/doc/camera_calibration_and_3d_reconstruction.html\#findfundamentalmat}{8-point-algorithm}\cite{hartley8point}
  from OpenCV, which uses RANSAC for more than 8 points. 2. We
  implemented
  \href{http://publica.fraunhofer.de/documents/N-246139.html}{\emph{Joint estimation of epipolar geometry and rectification parameters using point correspondences for stereoscopic TV sequences}} by Zilly, Mueller, Eisert and Kauff\cite{joint_estimation}.
	
  \includegraphics*[width=.92\textwidth]{10.jpg}
	
  If you choose the latter, you can calculate the rectifying
  homographies either according to the mentioned paper or using
  \href{http://docs.opencv.org/modules/calib3d/doc/camera_calibration_and_3d_reconstruction.html\#stereorectifyuncalibrated}{OpenCV's
  implementation of Hartley's algorithm}\cite{hartley}.
	
  \includegraphics*[width=.92\textwidth]{11.jpg}
	
\item
  Visualize the approximated fundamental matrix and homographies (main
  menu option 5). This will draw some epipolar lines on the first image
  pair loaded. Furthermore it will rectify theses images. The results
  are saved to the media folder.
	
  \includegraphics*[width=.92\textwidth]{12.jpg}
	
\item
  If you are satisfied with the results, you will have to copy the files
  homography\_CAM1.txt and homography\_CAM2.txt from the media folder to
  ARLibSamples/media/.
\end{enumerate}