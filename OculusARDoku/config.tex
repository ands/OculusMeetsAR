%\documentclass[fontsize=12pt, paper=a4]{report}
\documentclass[fontsize=12pt, paper=a4]{scrreprt}
%\documentclass[fontsize=12pt, paper=a4, headinclude, twoside=false, parskip=half+, pagesize=auto, numbers=noenddot, plainheadsepline, open=right]{scrreprt}
%\documentclass[11pt,a4paper]{article}
%\usepackage[latin1]{inputenc}
\usepackage[utf8]{inputenc}
\usepackage[T1]{fontenc}
%\usepackage[ngerman]{babel}
\usepackage[ngerman,english]{babel}
\usepackage[square,numbers]{natbib}
\usepackage{textcomp}
\usepackage{lmodern}

\usepackage[intlimits]{amsmath}
\usepackage{amssymb}
\usepackage{moreverb}

% PDF-Kompression
\pdfminorversion=5
\pdfobjcompresslevel=1

%\usepackage[automark]{scrpage2} % Kopf- und Fu�zeilen
%\usepackage{amsmath,marvosym} % Mathesachen
%\usepackage{mathpazo} % Palatino f�r Mathemodus
%\usepackage{mathpazo,tgpagella} % auch sehr sch�ne Schriften

\usepackage{setspace} % Zeilenabstand
\onehalfspacing % 1,5 Zeilen

% Schriften-Gr��en
\setkomafont{chapter}{\Large\rmfamily}
\setkomafont{section}{\large\rmfamily}
\setkomafont{subsection}{\normalsize\rmfamily}
\setkomafont{subsubsection}{\normalsize\rmfamily}
\setkomafont{paragraph}{\rmfamily}
\setkomafont{subparagraph}{\rmfamily}
%\setkomafont{chapterentry}{\large\rmfamily}
%\setkomafont{descriptionlabel}{\bfseries\rmfamily}
\setkomafont{captionlabel}{\upshape\bfseries}
\setkomafont{caption}{\itshape}

\usepackage[ngerman,english,pdfauthor={Andreas Mantler}, pdftitle={Design of a Semi-Actuated Tangible Device for Tabletop Interface Interactions}, breaklinks=true]{hyperref}
\usepackage[final]{microtype}
\usepackage{url}
\usepackage{multirow}
\usepackage{multicol}
\usepackage{tabularx}
\usepackage{longtable}
\usepackage{array}
\usepackage{float}
\usepackage{graphicx}
\usepackage{color}

\graphicspath{{images/}}
\DeclareGraphicsExtensions{.pdf,.png,.jpg}
\renewcommand{\thefigure}{\arabic{figure}}
\renewcommand{\thetable}{\arabic{table}}
\usepackage{subcaption}
\usepackage{rotating}
%\newcommand{\subfigureautorefname}{\figurename}
\usepackage[all]{hypcap}
%\setcapindent{0em}
%\setcapwidth[c]{0.9\textwidth}
%\setlength{\abovecaptionskip}{0.2cm}
\usepackage{chngcntr}
\counterwithout{figure}{chapter}

\usepackage{listings}
\usepackage{color}

\definecolor{mycomment}{rgb}{0,0.6,0}
\definecolor{mygray}{rgb}{0.5,0.5,0.5}
\definecolor{mymauve}{rgb}{0.58,0,0.22}

\lstset{ %
  backgroundcolor=\color{white},   % choose the background color; you must add \usepackage{color} or \usepackage{xcolor}
  basicstyle=\footnotesize,        % the size of the fonts that are used for the code
  breakatwhitespace=false,         % sets if automatic breaks should only happen at whitespace
  breaklines=true,                 % sets automatic line breaking
  commentstyle=\color{mycomment},  % comment style
  deletekeywords={...},            % if you want to delete keywords from the given language
  escapeinside={\%*}{*)},          % if you want to add LaTeX within your code
  extendedchars=true,              % lets you use non-ASCII characters; for 8-bits encodings only, does not work with UTF-8
  frame=single,                    % adds a frame around the code
  keepspaces=true,                 % keeps spaces in text, useful for keeping indentation of code (possibly needs columns=flexible)
  keywordstyle=\color{blue},       % keyword style
  language=C++,                    % the language of the code
%  otherkeywords={*,...},           % if you want to add more keywords to the set
  numbers=left,                    % where to put the line-numbers; possible values are (none, left, right)
  numbersep=5pt,                   % how far the line-numbers are from the code
  numberstyle=\tiny\color{mygray}, % the style that is used for the line-numbers
  rulecolor=\color{black},         % if not set, the frame-color may be changed on line-breaks within not-black text (e.g. comments (green here))
  showspaces=false,                % show spaces everywhere adding particular underscores; it overrides 'showstringspaces'
  showstringspaces=false,          % underline spaces within strings only
  showtabs=false,                  % show tabs within strings adding particular underscores
  stepnumber=1,                    % the step between two line-numbers. If it's 1, each line will be numbered
  stringstyle=\color{mymauve},     % string literal style
  tabsize=2,                       % sets default tabsize to 2 spaces
  title=\lstname                   % show the filename of files included with \lstinputlisting; also try caption instead of title
}
%\lstset{
%	extendedchars=true,
%	basicstyle=\tiny\ttfamily,
%	tabsize=2,
%	keywordstyle=\textbf,
%	commentstyle=\color{gray},
%	stringstyle=\textit,
%	numbers=left,
%	numberstyle=\tiny,
%	breakautoindent  = true,
%	breakindent      = 2em,
%	breaklines       = true,
%	postbreak        = ,
%	prebreak         = \raisebox{-.8ex}[0ex][0ex]{\Righttorque},
%}

% linksb�ndige Fu�boten
\deffootnote{1.5em}{1em}{\makebox[1.5em][l]{\thefootnotemark}}

\typearea{14}

\makeatletter
\newcommand{\saved@equation}{}
\let\saved@equation\equation
\def\equation{\@hyper@itemfalse\saved@equation}
\makeatother

% einr�cken nach �berschriften
\makeatletter
\let\@afterindentfalse\@afterindenttrue % Absatzeinzug auch nach �berschriften
\@afterindenttrue
\makeatother

\newcommand*\justify{%
  \fontdimen2\font=0.4em% interword space
  \fontdimen3\font=0.2em% interword stretch
  \fontdimen4\font=0.1em% interword shrink
  \fontdimen7\font=0.1em% extra space
  \hyphenchar\font=`\-% allowing hyphenation
}

% bild mit defnierter Breite einf�gen
\newcommand{\bild}[4]{
  \begin{figure}[!hbt]
    \centering
      \vspace{1ex}
      \includegraphics[width=#2]{images/#1}
      \caption[#4]{\label{img.#1} #3}
    \vspace{1ex}
  \end{figure}
}
% bild mit eigener Breite
\newcommand{\bilda}[3]{
  \begin{figure}[!hbt]
    \centering
      \vspace{1ex}
      \includegraphics{images/#1}
      \caption[#3]{\label{img.#1} #2}
      \vspace{1ex}
  \end{figure}
}

\font\texttrm = cmr17
\font\textttrm = cmr17 at 13pt

\addtolength{\textwidth}{-15mm}
\addtolength{\oddsidemargin}{13mm}

\addtolength{\headheight}{6.5mm}
\addtolength{\headsep}{0mm}%-24.5mm}
\addtolength{\textheight}{5mm}%28.5mm}
\addtolength{\footskip}{-7.5mm}

\usepackage[headsepline, footsepline, plainheadsepline]{scrpage2}
\pagestyle{scrheadings}
\clearscrheadfoot
\renewcommand*{\chaptermarkformat}{}
%\renewcommand*{\sectionmarkformat}{}
\automark[section]{chapter}
\renewcommand*{\chapterheadstartvskip}{\vspace{-11mm}}
\renewcommand*{\chapterheadendvskip}{\vspace{4mm}}

\setlength{\parindent}{0mm}

% punkte im Inhaltsverzeichnis
\makeatletter
\renewcommand*\l@chapter[2]{
  \ifnum \c@tocdepth >\m@ne
    \addpenalty{-\@highpenalty}
    \vskip 1.0em \@plus\p@
    \setlength\@tempdima{1.5em}
    \begingroup
      \parindent \z@ \rightskip \@pnumwidth
      \parfillskip -\@pnumwidth
      \leavevmode \bfseries
      \advance\leftskip\@tempdima
      \hskip -\leftskip
      #1\nobreak\normalfont\leaders\hbox{$\m@th
        \mkern \@dotsep mu\hbox{.}\mkern \@dotsep
        mu$}\hfill\nobreak\hb@xt@\@pnumwidth{\hss\bf #2}\par
      \penalty\@highpenalty
    \endgroup
  \fi}

%\renewcommand\l@section[2]{
%  \ifnum \c@tocdepth >\z@
%    \addpenalty\@secpenalty
%    \addvspace{1.0em \@plus\p@}
%    \setlength\@tempdima{1.5em}
%    \begingroup
%      \parindent \z@ \rightskip \@pnumwidth
%      \parfillskip -\@pnumwidth
%      \leavevmode \bfseries
%      \advance\leftskip\@tempdima
%      \hskip -\leftskip
%      #1\nobreak\ 
%      \leaders\hbox{$\m@th\mkern \@dotsep mu\hbox{.}\mkern \@dotsep mu$}
%     \hfil \nobreak\hb@xt@\@pnumwidth{\hss #2}\par
%    \endgroup
%  \fi}
%\makeatother

% newline after \paragraph
\makeatletter
\renewcommand\paragraph{%
   \@startsection{paragraph}{4}{0mm}%
      {-\baselineskip}%
      {.01\baselineskip}%
      {\normalfont\normalsize\bfseries}}
\makeatother

\AtBeginDocument{\addtocontents{toc}{\protect\thispagestyle{empty}}}

