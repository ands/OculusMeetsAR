This page explains the integration of postprocessing effects into the stereo output pipeline of ARLibOgre.

\section{GlowRenderTarget}\label{glowrendertarget}

As you can see in
\href{https://github.com/ands/OculusMeetsAR/blob/master/ARLib_Samples/RemoteGame/src/GlowRenderTarget.cpp}{GlowRenderTarget.cpp},
attaching
\hyperref[sec:ogre_library]{\texttt{RenderTargets}}
to the
\href{https://github.com/ands/OculusMeetsAR/blob/master/ARLib/include/ARLib/Ogre/RiftSceneNode.h}{\texttt{RiftSceneNode}}
can be intercepted (or decorated/wrapped as in the ``decorator
pattern'') by inserting a custom \texttt{RenderTarget} in between the
\texttt{RiftSceneNode} and the actual \texttt{RenderTarget}. In
\href{https://github.com/ands/OculusMeetsAR/blob/master/ARLib_Samples/RemoteGame/src/GlowRenderTarget.cpp}{GlowRenderTarget.cpp},
the latest viewports for both eye cameras are accessed after they were
attached to add a glow effect compositor.

\section{NPRWatercolorRenderTarget}\label{nprwatercolorrendertarget}

\href{https://github.com/ands/OculusMeetsAR/blob/master/ARLib_Samples/Common/src/NPRWatercolorRenderTarget.cpp}{NPRWatercolorRenderTarget.cpp}
demonstrates a more advanced example for postprocessing. Instead of just
directly rendering and compositing into the destination \texttt{RenderTarget},
the eye camera images are passed both blurred and unchanged to an OGRE
\texttt{MultiRenderTarget}. The blurred images are used to make voronoi-like
tilings of the input image (the brushstrokes), while edges detected in
the original images are subtracted from the voronoi tilings (sharp
outlines). The results are presented in separate scenes, which are seen
by another set of cameras. Those cameras are finally passed to the
destination \texttt{RenderTarget}.

\includegraphics*[width=.92\textwidth]{NPRWatercolorRenderTarget.png}

The whole process is a simplified version of a
\href{http://csnotes.upm.edu.my/kelasmaya/web.nsf/0/d82e709c4661ce464825766400365c4d/$FILE/p231-chen.pdf}{technique
by Chen, Turk and MacIntyre from 2008}\cite{watercolor}. The idea was that if both, real
images and rendered objects are postprocessed to look like stylized
watercolor paintings, it might be more difficult to tell them apart.